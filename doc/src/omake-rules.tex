%
%
%
\chapter{Rules}
\label{chapter:rules}
\cutname{omake-rules.html}

Rules are used by \OMake{} to specify how to build files.  At its simplest, a rule has the following
form.

\begin{verbatim}
    <target>: <dependencies>
        <commands>
\end{verbatim}

The \verb+<target>+ is the name of a file to be built.  The \verb+<dependencies>+ are a list of
files that are needed before the \verb+<target>+ can be built.  The \verb+<commands>+ are a list of
indented lines specifying commands to build the target.  For example, the following rule specifies
how to compile a file \verb+hello.c+.

\begin{verbatim}
    hello.o: hello.c
        $(CC) $(CFLAGS) -c -o hello.o hello.c
\end{verbatim}

This rule states that the \File{hello.o} file depends on the \File{hello.c} file.  If the
\File{hello.c} file has changed, the command \verb+$(CC) $(CFLAGS) -c -o hello.o hello.c+ is to
be executed to update the target file \verb+hello.o+.

A rule can have an arbitrary number of commands.  The individual command lines are executed
independently by the command shell.  The commands do not have to begin with a tab, but they must be
indented from the dependency line.

In addition to normal variables, the following special variables may be used in the body of a rule.

% We use \char* even for printable chars for consistency.
\begin{itemize}
\item \verb+$*+\index{\char36\char42}\index[var]{\char42}: the target name, without a suffix.
\item \verb+$@+\index{\char36\char64}\index[var]{\char64}: the target name.
\item \verb+$^+\index{\char36\char94}\index[var]{\char94}: a list of the sources, in alphabetical order, with
duplicates removed.
\item \verb.$+.\index{\char36\char43}\index[var]{\char43}: all the sources, in the original order.
\item \verb+$<+\index{\char36\char60}\index[var]{\char60}: the first source.
\end{itemize}

For example, the above \verb+hello.c+ rule may be simplified as follows.

\begin{verbatim}
    hello.o: hello.c
        $(CC) $(CFLAGS) -c -o $@ $<
\end{verbatim}

Unlike normal values, the variables in a rule body are expanded lazily, and binding is dynamic.  The
following function definition illustrates some of the issues.

\begin{verbatim}
    CLibrary(name, files) =
        OFILES = $(addsuffix .o, $(files))

        $(name).a: $(OFILES)
            $(AR) cq $@ $(OFILES)
\end{verbatim}

This function defines a rule to build a program called \verb+$(name)+ from a list of \verb+.o+
files.  The files in the argument are specified without a suffix, so the first line of the function
definition defines a variable \verb+OFILES+ that adds the \verb+.o+ suffix to each of the file
names.  The next step defines a rule to build a target library \verb+$(name).a+ from the
\verb+$(OFILES)+ files. The expression \verb+$(AR)+ is evaluated when the function is called, and
the value of the variable \verb+AR+ is taken from the caller's scope (see also the section on
Scoping).

\section{Implicit rules}
\index{rules, implicit}

Rules may also be implicit.  That is, the files may be specified by wildcard patterns.
The wildcard character is \verb+%+.  For example, the following rule specifies a default
rule for building \verb+.o+ files.

\begin{verbatim}
    %.o: %.c
        $(CC) $(CFLAGS) -c -o $@ $*.c
\end{verbatim}

This rule is a template for building an arbitrary \verb+.o+ file from
a \verb+.c+ file.

By default, implicit rules are only used for the targets in the current
directory. However subdirectories included via the \verb+.SUBDIRS+ rules
inherit all the implicit rules that are in scope (see also the section on
Scoping).

\section{Bounded implicit rules}
\index{rules, bounded implicit}

Implicit rules may specify the set of files they apply to.  The following syntax is used.

\begin{verbatim}
    <targets>: <pattern>: <dependencies>
        <commands>
\end{verbatim}

For example, the following rule applies only to the files \verb+a.o+ and \verb+b.o+.

\begin{verbatim}
   a.o b.o: %.o: %.c
        $(CC) $(CFLAGS) -DSPECIAL -c $*.c
\end{verbatim}

\section{section}
\formref{section}

Frequently, the commands in a rule body are expressions to be evaluated by the shell.  \Prog{omake}
also allows expressions to be evaluated by \Prog{omake} itself.

The syntax of these ``computed rules'' uses the \verb+section+ expression.  The following rule uses
the \Prog{omake} IO functions to produce the target \verb+hello.c+.

\begin{verbatim}
    hello.c:
        section
            FP = fopen(hello.c, w)
            fprintln($(FP), $""#include <stdio.h> int main() { printf("Hello world\n"); }"")
            close($(FP))
\end{verbatim}

This example uses the quotation \verb+$""...""+ (see also Section~\ref{section:quotes}) to quote the text being
printed.  These quotes are not included in the output file.  The \verb+fopen+, \verb+fprintln+, and
\verb+close+ functions perform file IO as discussed in the IO section.

In addition, commands that are function calls, or special expressions, are interpreted correctly.
Since the \verb+fprintln+ function can take a file directly, the above rule can be abbreviated as
follows.

\begin{verbatim}
    hello.c:
       fprintln($@, $""#include <stdio.h> int main() { printf("Hello world\n"); }"")
\end{verbatim}

\section{section rule}
\formref{section rule}

Rules can also be computed using the \verb+section rule+ form, where a rule body is expected instead
of an expression.  In the following rule, the file \verb+a.c+ is copied onto the \verb+hello.c+ file
if it exists, otherwise \verb+hello.c+ is created from the file \verb+default.c+.

\begin{verbatim}
    hello.c:
        section rule
           if $(target-exists a.c)
              hello.c: a.c
                 cat a.c > hello.c
           else
              hello.c: default.c
                 cp default.c hello.c
\end{verbatim}

In an implicit rule, any special variables, for example the one for the current target~(\verb+$@+)
or the first prerequisite~(\verb+$<+) are undefined; substitute \verb+$(file TARGET-PATTERN)+ and
\verb+$(file SOURCE-PATTERN)+:

\begin{verbatim}
    %.foo: %.bar
        section rule
            println($"target: $(file %.foo)")
            println($"prereq: $(file %.bar)")
            /usr/bin/printf "prerequisite was $(file %.bar)\n"  > $(file %.foo)
\end{verbatim}

However, if the \verb+section rule+ defines its own rule heads, all special variables work as usual
with respect to their rule heads:

\begin{verbatim}
    %.cmx: %.ml
        section rule
            if $(target-exists %.mli)
                %.cmx %.o: %.ml %.cmi :scanner: scan-ocaml-%.ml
                    $(OCamlOpt) -c $<
            elseif $(BYTE_ENABLED)
                %.cmx %.cmi %.o %.cmo: %.ml :scanner: scan-ocaml-%.ml
                    $(OCamlC) -c $<
                    $(OCamlOpt) -c $<
            else
                %.cmx %.cmi %.o: %.ml :scanner: scan-ocaml-%.ml
                    $(OCamlOpt) -c $<
\end{verbatim}

\section{Special dependencies}
\index{rule, options}

\subsection{:exists:}
\index{:exists:}

In some cases, the contents of a dependency do not matter, only whether the file exists or not.  In
this case, the \verb+:exists:+ qualifier can be used for the dependency.

\begin{verbatim}
    foo.c: a.c :exists: .flag
       if $(test -e .flag)
           $(CP) a.c $@
\end{verbatim}

\subsection{:effects:}
\index{:effects:}

Some commands produce files by side-effect.  For example, the
\Cmd{latex}{1} command produces a \verb+.aux+ file as a side-effect of
producing a \verb+.dvi+ file.  In this case, the \verb+:effects:+
qualifier can be used to list the side-effect explicitly.
\Prog{omake} is careful to avoid simultaneously running programs that
have overlapping side-effects.

\begin{verbatim}
    paper.dvi: paper.tex :effects: paper.aux
        latex paper
\end{verbatim}

\subsection{:value:}
\index{:value:}

The \verb+:value:+ dependency is used to specify that the rule execution depends on the value of an
expression.  For example, the following rule

\begin{verbatim}
    a: b c :value: $(X)
        ...
\end{verbatim}

specifies that ``a'' should be recompiled if the value of \verb+$(X)+ changes
(X does not have to be a filename).  This is intended to allow greater
control over dependencies.

In addition, it can be used instead of other kinds of dependencies. For example,
the following rule:

\begin{verbatim}
    a: b :exists: c
        commands
\end{verbatim}

is the same as

\begin{verbatim}
    a: b :value: $(target-exists c)
        commands
\end{verbatim}

Notes:
\begin{itemize}
\item The values are arbitrary (they are not limited to variables)
\item The values are evaluated at rule expansion time, so expressions
containing variables like \verb+$@+, \verb+$^+, etc are legal.
\end{itemize}

\section{\code{.SCANNER} rules}
\targetref{.SCANNER}

Scanner rules define a way to specify automatic dependency scanning.  A \verb+.SCANNER+ rule has the
following form.

\begin{verbatim}
    .SCANNER: target: dependencies
        commands
\end{verbatim}

The rule is used to compute additional dependencies that might be defined in the source files for
the specified target. The result of executing the scanner commands \emph{must} be a sequence of
dependencies in OMake format, printed to the standard output.  For example, on GNU systems the
\verb+gcc -MM foo.c+ produces dependencies for the file \verb+foo.c+ (based on \verb+#include+
information).

We can use this to specify a scanner for C files that adds the scanned dependencies for the
\verb+.o+ file.  The following scanner specifies that dependencies for a file, say \verb+foo.o+ can
be computed by running \verb+gcc -MM foo.c+.  Furthermore, \verb+foo.c+ is a dependency, so the
scanner should be recomputed whenever the \verb+foo.c+ file changes.

\begin{verbatim}
    .SCANNER: %.o: %.c
        gcc -MM $<
\end{verbatim}

Let's suppose that the command \verb+gcc -MM foo.c+ prints the following line.

\begin{verbatim}
    foo.o: foo.h /usr/include/stdio.h
\end{verbatim}

The result is that the files \verb+foo.h+ and \verb+/usr/include/stdio.h+ are considered to be
dependencies of \verb+foo.o+---that is, \verb+foo.o+ should be rebuilt if either of these files
changes.

This works, to an extent.  One nice feature is that the scanner will be re-run whenever the
\verb+foo.c+ file changes.  However, one problem is that dependencies in C are \emph{recursive}.
That is, if the file \verb+foo.h+ is modified, it might include other files, establishing further
dependencies.  What we need is to re-run the scanner if \verb+foo.h+ changes too.

We can do this with a \emph{value} dependency.  The variable \verb+$&+ is defined as the dependency
results from any previous scan.  We can add these as dependencies using the \verb+digest+ function,
which computes an MD5 digest of the files.

\begin{verbatim}
    .SCANNER: %.o: %.c :value: $(digest $&)
        gcc -MM $<
\end{verbatim}

Now, when the file \verb+foo.h+ changes, its digest will also change, and the scanner will be re-run
because of the value dependency (since \verb+$&+ will include \verb+foo.h+).

This still is not quite right.  The problem is that the C compiler uses a \emph{search-path} for
include files.  There may be several versions of the file \verb+foo.h+, and the one that is chosen
depends on the include path.  What we need is to base the dependencies on the search path.

The \verb+$(digest-in-path-optional ...)+ function computes the digest based on a search path,
giving us a solution that works.

\begin{verbatim}
    .SCANNER: %.o: %.c :value: $(digest-in-path-optional $(INCLUDES), $&)
       gcc -MM $(addprefix -I, $(INCLUDES)) $<
\end{verbatim}

The standard output of the scanner rules will be captured by OMake and is not allowed to contain any
content that OMake will not be able to parse as a dependency. The output is allowed to contain
dependency specifications for unrelated targets, however such dependencies will be ignored. The
scanner rules are allowed to produce arbitrary output on the standard error channel --- such output
will be handled in the same way as the output of the ordinary rules (in other words, it will be
presented to the user, when dictated by the \verb+--output-+$\ldots$ options enabled).

Additional examples of the \verb+.SCANNER+ rules can be found in Section~\ref{section:scanner-exm}.

\subsection{Named scanners, and the \code{:scanner:} dependencies}
\index{:scanner:}

Sometimes it may be useful to specify explicitly which scanner should be used in a rule.  For
example, we might compile \verb+.c+ files with different options, or (heaven help us) we may be
using both \verb+gcc+ and the Microsoft Visual C++ compiler \verb+cl+. In general, the target of a
\verb+.SCANNER+ is not tied to a particular target, and we may name it as we like.

\begin{verbatim}
    .SCANNER: scan-gcc-%.c: %.c :value: $(digest-in-path-optional $(INCLUDES), $&)
        gcc -MM $(addprefix -I, $(INCLUDES)) $<

    .SCANNER: scan-cl-%.c: %.c :value: $(digest-in-path-optional $(INCLUDES), $&)
        cl --scan-dependencies-or-something $(addprefix /I, $(INCLUDES)) $<
\end{verbatim}

The next step is to define explicit scanner dependencies.  The \verb+:scanner:+ dependency is used
for this.  In this case, the scanner dependencies are specified explicitly.

\begin{verbatim}
    $(GCC_FILES): %.o: %.c :scanner: scan-gcc-%.c
        gcc ...

    $(CL_FILES): %.obj: %.c :scanner: scan-cl-%.c
        cl ...
\end{verbatim}

Explicit \verb+:scanner:+ scanner specification may also be used to state that a single
\verb+.SCANNER+ rule should be used to generate dependencies for more than one target. For example,

\begin{verbatim}
    .SCANNER: scan-all-c: $(GCC_FILES) :value: $(digest-in-path-optional $(INCLUDES), $&)
        gcc -MM $(addprefix -I, $(INCLUDES)) $(GCC_FILES)

    $(GCC_FILES): %.o: %.c :scanner: scan-all-c
        ...
\end{verbatim}

The above has the advantage of only running \Prog{gcc} once and a disadvantage that when a single
source file changes, all the files will end up being re-scanned.

\subsection{Notes}

In most cases, you won't need to define scanners of your own.  The standard installation includes
default scanners (both explicitly and implicitly named ones) for C, OCaml, and \LaTeX{} files.

The \hypervarx{SCANNER_MODE}{SCANNER\_MODE} controls the usage of implicit scanner dependencies.

The explicit \verb+:scanner:+ dependencies reduce the chances of scanner mis-specifications. In
large complicated projects it might be a good idea to set \verb+SCANNER_MODE+ to \verb+error+ and
use only the named \verb+.SCANNER+ rules and explicit \verb+:scanner:+ specifications.

\section{.DEFAULT}
\targetref{.DEFAULT}

The \verb+.DEFAULT+ target specifies a target to be built by default
if \Prog{omake} is run without explicit targets.  The following rule
instructs \Prog{omake} to build the program \verb+hello+ by default

\begin{verbatim}
   .DEFAULT: hello
\end{verbatim}

\section{.SUBDIRS}
\targetref{.SUBDIRS}

The \verb+.SUBDIRS+ target is used to specify a set of subdirectories
that are part of the project.  Each subdirectory should have its own
\File{OMakefile}, which is evaluated in the context of the current
environment.

\begin{verbatim}
   .SUBDIRS: src doc tests
\end{verbatim}

This rule specifies that the \verb+OMakefile+s in each of the \verb+src+, \verb+doc+, and
\verb+tests+ directories should be read.

In some cases, especially when the \verb+OMakefile+s are very similar in a large number of
subdirectories, it is inconvenient to have a separate \verb+OMakefile+ for each directory.  If the
\verb+.SUBDIRS+ rule has a body, the body is used instead of the \verb+OMakefile+.

\begin{verbatim}
   .SUBDIRS: src1 src2 src3
      println(Subdirectory $(CWD))
      .DEFAULT: lib.a
\end{verbatim}

In this case, the \verb+src1+, \verb+src2+, and \verb+src3+ files do not need \verb+OMakefile+s.
Furthermore, if one exists, it is ignored.  The following includes the file if it exists.

\begin{verbatim}
   .SUBDIRS: src1 src2 src3
       if $(file-exists OMakefile)
          include OMakefile
       .DEFAULT: lib.a
\end{verbatim}

\section{.INCLUDE}
\targetref{.INCLUDE}

The \verb+.INCLUDE+ target is like the \verb+include+ directive, but it specifies a rule to build
the file if it does not exist.

\begin{verbatim}
   .INCLUDE: config
       echo "CONFIG_READ = true" > config

    echo CONFIG_READ is $(CONFIG_READ)
\end{verbatim}

You may also specify dependencies to an \verb+.INCLUDE+ rule.

\begin{verbatim}
   .INCLUDE: config: config.defaults
      cp config.defaults config
\end{verbatim}

A word of caution is in order here.  The usual policy is used for determining when the rule is
out-of-date.  The rule is executed if any of the following hold.

\begin{itemize}
\item the target does not exist,
\item the rule has never been executed before,
\item any of the following have changed since the last time the rule was executed,
\begin{itemize}
\item the target,
\item the dependencies,
\item the commands-text.
\end{itemize}
\end{itemize}

In some of the cases, this will mean that the rule is executed even if the target file already
exists.  If the target is a file that you expect to edit by hand (and therefore you don't want to
overwrite it), you should make the rule evaluation conditional on whether the target already exists.

\begin{verbatim}
   .INCLUDE: config: config.defaults
       # Don't overwrite my carefully hand-edited file
       if $(not $(file-exists config))
          cp config.defaults config
\end{verbatim}

\section{.PHONY}
\targetref{.PHONY}

A ``phony'' target is a target that is not a real file, but exists to collect a set of dependencies.
Phony targets are specified with the \verb+.PHONY+ rule.  In the following example, the
\verb+install+ target does not correspond to a file, but it corresponds to some commands that should
be run whenever the \verb+install+ target is built (for example, by running \verb+omake install+).

\begin{verbatim}
   .PHONY: install

   install: myprogram.exe
      cp myprogram.exe /usr/bin
\end{verbatim}

\section{Rule scoping}
\index{rule, scoping}

As we have mentioned before, \Prog{omake} is a \emph{scoped} language.  This provides great
flexibility---different parts of the project can define different configurations without interfering
with one another (for example, one part of the project might be compiled with \verb+CFLAGS=-O3+ and
another with \verb+CFLAGS=-g+).

But how is the scope for a target file selected?  Suppose we are building a file \verb+dir/foo.o+.
\Prog{omake} uses the following rules to determine the scope.

\begin{itemize}
\item First, if there is an \emph{explicit} rule for building \verb+dir/foo.o+ (a rule with no
wildcards), the context for that rule determines the scope for building the target.
\item Otherwise, the directory \verb+dir/+ must be part of the project.  This normally means that a
configuration file \verb+dir/OMakefile+ exists (although, see the \verb+.SUBDIRS+ section for
another way to specify the \verb+OMakefile+).  In this case, the scope of the target is the scope at
the end of the \verb+dir/OMakefile+.
\end{itemize}

To illustrate rule scoping, let's go back to the example of a ``Hello world'' program with two
files.  Here is an example \verb+OMakefile+ (the two definitions of \verb+CFLAGS+ are for
illustration).

\begin{verbatim}
    # The executable is compiled with debugging
    CFLAGS = -g
    hello: hello_code.o hello_lib.o
       $(CC) $(CFLAGS) -o $@ $+

    # Redefine CFLAGS
    CFLAGS += -O3
\end{verbatim}

In this project, the target \verb+hello+ is \emph{explicit}.  The scope of the \verb+hello+ target
is the line beginning with \verb+hello:+, where the value of \verb+CFLAGS+ is \verb+-g+.  The other
two targets, \verb+hello_code.o+ and \verb+hello_lib.o+ do not appear as explicit targets, so their
scope is at the end of the \verb+OMakefile+, where the \verb+CFLAGS+ variable is defined to be
\verb+-g -O3+.  That is, \verb+hello+ will be linked with \verb+CFLAGS=-g+ and the \verb+.o+ files
will be compiled with \verb+CFLAGS=-g -O3+.

We can change this behavior for any of the targets by specifying them as explicit targets.  For
example, suppose we wish to compile \verb+hello_lib.o+ with a preprocessor variable \verb+LIBRARY+.

\begin{verbatim}
    # The executable is compiled with debugging
    CFLAGS = -g
    hello: hello_code.o hello_lib.o
       $(CC) $(CFLAGS) -o $@ $+

    # Compile hello_lib.o with CFLAGS = -g -DLIBRARY
    section
        CFLAGS += -DLIBRARY
        hello_lib.o:

    # Redefine CFLAGS
    CFLAGS += -O3
\end{verbatim}

In this case, \verb+hello_lib.o+ is also mentioned as an explicit target, in a scope where
\verb+CFLAGS=-g -DLIBRARY+.  Since no rule body is specified, it is compiled using the usual
implicit rule for building \verb+.o+ files (in a context where \verb+CFLAGS=-g -DLIBRARY+).

\subsection{Scoping of implicit rules}
\label{section:implicit-scoping}

Implicit rules (rules containing wildcard patterns) are \emph{not} global, they follow the normal
scoping convention.  This allows different parts of a project to have different sets of implicit
rules.  If we like, we can modify the example above to provide a new implicit rule for building
\verb+hello_lib.o+.

\begin{verbatim}
    # The executable is compiled with debugging
    CFLAGS = -g
    hello: hello_code.o hello_lib.o
       $(CC) $(CFLAGS) -o $@ $+

    # Compile hello_lib.o with CFLAGS = -g -DLIBRARY
    section
        %.o: %.c
            $(CC) $(CFLAGS) -DLIBRARY -c $<
        hello_lib.o:

    # Redefine CFLAGS
    CFLAGS += -O3
\end{verbatim}

In this case, the target \verb+hello_lib.o+ is built in a scope with a new implicit rule for
building \verb+%.o+ files.  The implicit rule adds the \verb+-DLIBRARY+ option.  This implicit rule
is defined only for the target \verb+hello_lib.o+; the target \verb+hello_code.o+ is built as
normal.

\subsection{Scoping of \code{.SCANNER} rules}
\index{.SCANNER}\index[target]{.SCANNER}

Scanner rules are scoped the same way as normal rules.  If the \verb+.SCANNER+ rule is explicit
(containing no wildcard patterns), then the scope of the scan target is the same as the the rule.
If the \verb+.SCANNER+ rule is implicit, then the environment is taken from the \verb+:scanner:+
dependency.

\begin{verbatim}
    # The executable is compiled with debugging
    CFLAGS = -g
    hello: hello_code.o hello_lib.o
       $(CC) $(CFLAGS) -o $@ $+

    # scanner for .c files
    .SCANNER: scan-c-%.c: %.c
       $(CC) $(CFLAGS) -MM $<

    # Compile hello_lib.o with CFLAGS = -g -DLIBRARY
    section
        CFLAGS += -DLIBRARY
        hello_lib.o: hello_lib.c :scanner: scan-c-hello_lib.c
           $(CC) $(CFLAGS) -c $<

    # Compile hello_code.c with CFLAGS = -g -O3
    section
        CFLAGS += -O3
        hello_code.o: hello_code.c :scanner: scan-c-hello_code.c
           $(CC) $(CFLAGS) -c $<
\end{verbatim}

Again, this is for illustration---it is unlikely you would need to write a complicated configuration
like this!  In this case, the \verb+.SCANNER+ rule specifies that the C-compiler should be called
with the \verb+-MM+ flag to compute dependencies.  For the target \verb+hello_lib.o+, the scanner
is called with \verb+CFLAGS=-g -DLIBRARY+, and for \verb+hello_code.o+ it is called with
\verb+CFLAGS=-g -O3+.

\subsection{Scoping for \code{.PHONY} targets}
\label{section:PHONY-scoping}\index{.PHONY}\index[target]{.PHONY}

Phony targets (targets that do not correspond to files) are defined with a \verb+.PHONY:+ rule.
Phony targets are scoped as usual.  The following illustrates a common mistake, where the
\verb+.PHONY+ target is declared \emph{after} it is used.

\begin{verbatim}
    # !!This example is broken!!
    all: hello

    hello: hello_code.o hello_lib.o
        $(CC) $(CFLAGS) -o $@ $+

    .PHONY: all
\end{verbatim}

This doesn't work as expected because the \verb+.PHONY+ declaration occurs too late.  The proper way
to write this example is to place the \verb+.PHONY+ declaration first.

\begin{verbatim}
    # Phony targets must be declared before being used
    .PHONY: all

    all: hello

    hello: hello_code.o hello_lib.o
        $(CC) $(CFLAGS) -o $@ $+
\end{verbatim}

Phony targets are passed to subdirectories.  As a practical matter, it is wise to declare all
\verb+.PHONY+ targets in your root \verb+OMakefile+, before any \verb+.SUBDIRS+.  This will ensure
that 1) they are considered as phony targets in each of the subdirectories, and 2) you can build them
from the project root.

\begin{verbatim}
    .PHONY: all install clean

    .SUBDIRS: src lib clib
\end{verbatim}

Note that when a \verb+.PHONY+ target is inherited by a subdirectory via a \verb+.SUBDIRS+, a whole
hierarchy of \verb+.PHONY+ targets (that are a part of the global one) is created, as described in
Section~\ref{section:PHONY-hierarchy} below.

\section{Running \OMake{} from a subdirectory}
\label{section:running-from-subdir}
Running \verb+omake foo+ asks \OMake{} to build the file \verb+foo+ in context of the \emph{whole}
project, even when running from a subdirectory of the project. Therefore, if \verb+bar/baz+ is a
regular target (not a \verb+.PHONY+ one), then running \verb+omake bar/baz+ and running
\verb+(cd bar; omake baz)+ are usually equivalent.

There are two noteworthy exceptions to the above rule:
\begin{itemize}
\item If the subdirectory is not a part of the project (there is no \verb+.SUBDIRS+) for it, then
\OMake{} will complain if you try to run it in that directory.
\item If a subdirectory contains an \verb+OMakeroot+ of its own, this would designate
the subdirectory as a separate project (which is usually a bad idea and is not recommended).
\end{itemize}

\subsection{Phony targets in a subdirectory}
\label{section:PHONY-in-subdirs}\index{.PHONY}\index[target]{.PHONY}\index{.DEFAULT}\index[target]{.DEFAULT}

Suppose you have a \verb+.PHONY: clean+ declared in your root \verb+OMakefile+ and
both the root \verb+OMakefile+ and the \verb+OMakefile+ in some of the subdirectories contain
\verb+clean:+ rules.  In this case
\begin{itemize}
\item Running \verb+omake clean+ in the root directory will execute all the rules (each in the
appropriate directory);
\item Running \verb+omake clean+ in the subdirectory will execute just its local one, as well as the
ones from the subdirectories of the current directory.
\end{itemize}

The above equally applies to the built-in \verb+.PHONY+ targets, including \hypertargn{.DEFAULT}.
Namely, if \OMake{} is executed (without argument) in the root directory of a project, all the
\verb+.DEFAULT+ targets in the project will be built. On the other hand, when \OMake{} is executed
(without argument) in a subdirectory, only the \verb+.DEFAULT+ targets defined in and under that
subdirectory will be built.

The following Section explains the underlying semantics that gives rise to the above behavior.

\subsection{Hierarchy of \code{.PHONY} targets}
\label{section:PHONY-hierarchy}\index{.PHONY}\index[target]{.PHONY}

When the the root \verb+OMakefile+ contains a \verb+.PHONY: clean+ directive, it creates:
\begin{itemize}
\item A ``global'' phony target \verb+/.PHONY/clean+ (note the leading ``\verb+/+'');
\item A ``relative'' phony target attached to the current directory --- \verb+.PHONY/clean+ (note
the lack of the leading ``\verb+/+'');
\item A dependency \verb+/.PHONY/clean: .PHONY/clean+.
\end{itemize}

All the \verb+clean: ...+ rules in the root \verb+OMakefile+ following this \verb+.PHONY: clean+
declaration would be interpreted as rules for the \verb+.PHONY/clean+ target.

Now when \OMake{} then comes across a \verb+.SUBDIRS: foo+ directive (when it is in scope of the above
\verb+.PHONY: clean+ declaration), it does the following:
\begin{itemize}
\item Creates a new \verb+.PHONY/foo/clean+ ``relative'' phony target;
\item Creates the dependency \verb+.PHONY/clean: .PHONY/foo/clean+;
\item Processes the body of the \verb+.SUBDIRS: foo+ directive, or reads the \verb+foo/OMakefile+
file, if the body is empty. While doing that, it interprets its instructions relative to the
\verb+foo+ directory. In particular, all the \verb+clean: ...+ rules will be taken to apply to
\verb+.PHONY/foo/clean+.
\end{itemize}

Now when you run \verb+omake clean+ in the root directory of the project, it is interpreted as
\verb+omake .PHONY/clean+ (similar to how it happens with the normal targets), so both the rules for
\verb+.PHONY/clean+ are executed and the rules for its dependency 
\verb+.PHONY/foo/clean+. Running \verb+(cd foo; omake clean)+ is, as for normal targets, equivalent to running
\verb+omake .PHONY/foo/clean+ and only those rules that apply to \verb+.PHONY/foo/clean+ will be executed.

\section{Pathnames in rules}

In rules, the targets and dependencies are first translated to \emph{file} values (as in the
\hyperfun{file}).  They are then translated to strings for the command line.
This can cause some unexpected behavior.  In the following example, the \hyperfun{absname}
is the absolute pathname for the file \verb+a+, but the rule still prints
the relative pathname.

\begin{verbatim}
    .PHONY: demo
    demo: $(absname a)
        echo $<

    # omake demo
    a
\end{verbatim}

There is arguably a good reason for this.  On Win32 systems, the \verb+/+ character is viewed as an
``option specifier.''  The pathname separator is the \verb+\+ character.  \OMake{} translates the
filenames automatically so that things work as expected on both systems.

\begin{verbatim}
   demo: a/b
       echo $<

   # omake demo (on a Unix system)
   a/b
   # omake demo (on a Win32 system)
   a\b
\end{verbatim}

Sometimes you may wish that target strings to be passed literally to the commands in the rule.
One way to do this is to specify them literally.

\begin{verbatim}
    SRC = a/b $(absname c/d)
    demo: $(SRC)
        echo $(SRC)

    # omake demo (on a Win32 system)
    a/b c:\...\c\d
\end{verbatim}

Alternately, you might wish that filenames be automatically expanded to absolute pathnames.  For
example, this might be useful when parsing the \OMake{} output to look for errors.  For this, you can
use the \verb+--absname+ option (Section~\ref{option:--absname}).  If you call \verb+omake+ with the
\verb+--absname+ option, all filenames will be expanded to absolute names.

\begin{verbatim}
    # omake --absname demo (on a Unix system)
    /home/.../a/b /home/.../c/d
\end{verbatim}

Alternately, the \verb+--absname+ option is scoped.  If you want to use it for only a few rules, you
can use the \hyperfun{OMakeFlags} to control how it is applied.
    
\begin{verbatim}
   section
      OMakeFlags(--absname)
      demo: a
          echo $<

   # omake demo
   /home/.../a
\end{verbatim}

\textbf{N.B.} The \verb+--absname+ option is currently an experimental feature.

% -*-
% Local Variables:
% Mode: LaTeX
% fill-column: 100
% TeX-master: "paper"
% TeX-command-default: "LaTeX/dvips Interactive"
% End:
% vim:tw=100:fo=tcq:
% -*-
